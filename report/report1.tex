\documentclass{article}
\usepackage{amsmath}
\usepackage{xeCJK}
\usepackage{fontspec}

\setCJKmainfont{SimSun} 

\title{电机PID控制及其GUI界面设计}
\author{杜佳颖}
\date{\today}

\begin{document}
\maketitle

\section{PID部分}
在本项目中,我们使用PID控制算法来调节电机的转速。PID控制器的设置需要考虑多个方面,包括参数设置、采样处理和防抖等。

\subsection{PID参数设置}
我们首先使用一般的PID控制算法,通过不断调整P、I、D参数,得到了一个较优的PID参数组合。但是,即使在最优参数下,电机的转速仍然无法跟上手动操纵杆的速度。因此,我们在控制逻辑上进行了优化:当电机位置接近设定目标时,使用PID控制;在其他情况下,使用PWM最大值255进行控制。这样的分段控制方法可以确保电机在大幅度调整时快速响应,同时在接近目标时精确调整。

\subsection{采样处理}
为了确保PID控制的精度,我们在代码中引入了采样处理。通过读取传感器的值,计算出当前电机位置与设定值之间的误差,并根据误差调整控制信号。每次更新控制信号时,我们都会记录当前时间,并在下一次计算时使用时间差来调整积分和微分项,确保控制信号的稳定性和准确性。

\subsection{防抖处理}
在实际应用中,传感器读数可能会因为环境噪声或机械振动而产生抖动。为了避免这些干扰影响控制效果,我们在计算误差的过程中加入了防抖处理。具体做法是,当误差的变化量小于一定阈值时,不更新微分项,从而减少因抖动引起的控制信号波动。

以下是主要代码段的概括:
- 初始化PID参数和各类引脚。
- 在主循环中,不断读取传感器和设定值,计算误差并根据误差大小选择PID控制或PWM最大值控制。
- 通过串口监视器进行实时参数调整,方便调试和优化。

\section{Python GUI部分}
为了方便用户调整PID参数和监控电机状态,我们设计了一个基于Python的GUI界面。通过串口通讯,GUI界面可以实时发送参数调整命令,并接收电机的运行状态。

\subsection{界面设计}
GUI界面使用Tkinter库实现,包含以下功能模块:
- 串口选择和连接按钮。
- 用于调整Kp、Ki、Kd参数的滑动条和设置按钮。
- 设定值输入框和设置按钮。
- 显示当前电机位置和设定值的标签。
- LED状态显示。

\subsection{串口通讯}
GUI通过串口与电机控制器进行通讯。用户在GUI上调整参数时,相应命令会通过串口发送到控制器。控制器实时反馈电机位置和状态,GUI界面相应更新显示。为了确保通讯的可靠性,我们在程序中加入了多线程处理,保证主界面不会因为串口读写操作而卡顿。

\section{发挥部分EXE方案}
由于时间和资源限制,目前我们还没有开发独立的EXE应用程序。未来计划将Python GUI部分打包为独立的EXE文件,方便用户在无需安装Python环境的情况下直接使用。

\end{document}
